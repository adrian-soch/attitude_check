\documentclass[12pt, titlepage]{article}

\usepackage{booktabs}
\usepackage{tabularx}
\usepackage{hyperref}
\usepackage{float}
\hypersetup{ colorlinks, citecolor=black, filecolor=black, linkcolor=red, urlcolor=blue }
\usepackage[round]{natbib}

%% Comments

\usepackage{color}

% \newif\ifcomments\commentstrue %displays comments
\newif\ifcomments\commentsfalse %so that comments do not display

\ifcomments
\newcommand{\authornote}[3]{\textcolor{#1}{[#3 ---#2]}}
\newcommand{\todo}[1]{\textcolor{red}{[TODO: #1]}}
\else
\newcommand{\authornote}[3]{}
\newcommand{\todo}[1]{}
\fi

\newcommand{\wss}[1]{\authornote{blue}{SS}{#1}} 
\newcommand{\plt}[1]{\authornote{magenta}{TPLT}{#1}} %For explanation of the template
\newcommand{\an}[1]{\authornote{cyan}{Author}{#1}}

%% Common Parts

\newcommand{\progname}{ProgName} % PUT YOUR PROGRAM NAME HERE
\newcommand{\authname}{Adrian Sochaniwsky} % AUTHOR NAMES                  

\usepackage{hyperref}
    \hypersetup{colorlinks=true, linkcolor=blue, citecolor=blue, filecolor=blue,
                urlcolor=blue, unicode=false}
    \urlstyle{same}
                                


\begin{document}

\title{Verification and Validation Report: \progname}
\author{\authname}
\date{\today}

\maketitle

\pagenumbering{roman}

\section{Revision History}

\begin{tabularx}{\textwidth}{p{3cm}p{2cm}X} \toprule {\bf Date} & {\bf Version} & {\bf Notes}\\
\midrule
April 5, 2024 & 1.0 & Initial version\\
\bottomrule
\end{tabularx}

~\newpage

\section{Symbols, Abbreviations and Acronyms}

\renewcommand{\arraystretch}{1.2}
\begin{tabular}{l l}
  \toprule
  \textbf{symbol} & \textbf{description}\\
  \midrule
  T & Test\\
  UT & Unit Test \\
  \bottomrule
\end{tabular}\\

\wss{symbols, abbreviations or acronyms -- you can reference the SRS tables if needed}

\newpage

\tableofcontents

\listoftables %if appropriate

% \listoffigures %if appropriate

\newpage

\pagenumbering{arabic}

This document reports the results of executing the
\href{https://github.com/adrian-soch/attitude_check/blob/main/docs/VnVPlan/VnVPlan.pdf}{VnV Plan}.

\section{Functional Requirements Evaluation}

This section covers the evaluation of the functional requirements.

\subsection{T1} This test evaluates the input and output requirements of \progname{}. This test is
implemented by the first 3 unit tests in
\url{https://github.com/adrian-soch/attitude_check/blob/main/test/attitude_check_test.cpp}. This
requirement is \textbf{satisfied} by \progname{}.

\subsection{T2}\label{t2}

This test evaluates the calculations when accel, gyro, and mag data is provided. To conduct this
test the following steps are required:
\begin{enumerate}
  \item Create a csv file with sensor data and the ground truth with
  \href{https://github.com/adrian-soch/attitude_check/blob/main/scripts/create_sensor_csv.py}{create\_sensor\_csv.py}
  \item Use
  \href{https://github.com/adrian-soch/attitude_check/blob/main/examples/basic_orientation_calculation.cpp}{basic\_orientation\_calculation.cpp}
  to estimate quaternions.
  \item Calculate RMSE for the inclination component of the quaternion using
  \href{https://github.com/adrian-soch/attitude_check/blob/main/scripts/compare_results.py}{compare\_results.py}
\end{enumerate}

\begin{table}[H]
    \centering
    \caption{Accel, Gyro, Mag Test Results}
    \label{tab:my-table}
    \begin{tabular}{|l|l|}
    \hline
    Program        & Inclination RMSE (deg) \\ \hline
    Attitude Check & 7.027 deg              \\ \hline
    \href{https://ahrs.readthedocs.io/en/latest/filters/madgwick.html}{AHRS} & 7.218 deg \\ \hline
    \hline
    \% Difference  & 2.65 \%                \\ \hline
    \end{tabular}
\end{table}

Since $2.65 < 3$ this test passes. Furthermore, \progname{} is more accurate than the AHRS implementation for this test data.

\subsection{T3} This test evaluates the calculations when accel and gyro data is provided. It is
conducted using the procedure from T2.

\begin{table}[H]
    \centering
    \caption{Accel and Gyro Test Results}
    \label{tab:my-table}
    \begin{tabular}{|l|l|}
    \hline
    Program        & Inclination RMSE (deg) \\ \hline
    Attitude Check & 5.9709                 \\ \hline
    \href{https://ahrs.readthedocs.io/en/latest/filters/madgwick.html}{AHRS} & 5.9708 \\ \hline
    \% Difference  & 0.002 \%               \\ \hline
    \end{tabular}
  \end{table}

  Since $0.002 < 3$ this test passes.

\section{Nonfunctional Requirements Evaluation}

This section covers the evaluation of the nonfunctional requirements.

\subsection{Accuracy} \label{acc}

See T2 and T3 for accuracy results.

\subsection{Understandability}

This NFR was outside the scope of the VnV Plan.

\subsection{Performance}

The runtime for the SparkFun Razor 9DoF IMU are reported:

\begin{itemize}
  \item (Acc, Gyr, Mag) estimation: 1.5 ms
  \item (Acc, Gyr) estimation: 0.8 ms
\end{itemize}

Both are fast enough to keep up with average sensor update rates (20 - 600 Hz).

\subsection{Maintainability}

A likely change was not implemented, thus this test was not executed.

\subsection{Portability}

\progname{} was successfully executed on x86 and ARM hardware, it compiles for Linux and Arduino.
Thus, it passes the portability test.

\section{Unit Testing}

Each file with source code has its own unit test file
(\url{https://github.com/adrian-soch/attitude_check/tree/main/test}). Each commit tot the main branch must pass all unit tests. There are 36 unit tests in total, covering 100\% of the code.

Each unit test and its result are below:

\begin{small}
\begin{verbatim}
  Test project /home/adrian/dev/attitude_check/build/test
      Start  1: ACheck_Test_Fixture.invalid_init
 1/36 Test  #1: ACheck_Test_Fixture.invalid_init ........................  Passed
      Start  2: ACheck_Test_Fixture.invalid_update
 2/36 Test  #2: ACheck_Test_Fixture.invalid_update ......................  Passed
      Start  3: ACheck_Test_Fixture.marg_zero_gyro
 3/36 Test  #3: ACheck_Test_Fixture.marg_zero_gyro ......................  Passed
      Start  4: ACheck_Test_Fixture.marg_zero_mag
 4/36 Test  #4: ACheck_Test_Fixture.marg_zero_mag .......................  Passed
      Start  5: ACheck_Test_Fixture.marg_zero_acc
 5/36 Test  #5: ACheck_Test_Fixture.marg_zero_acc .......................  Passed
      Start  6: ACheck_Test_Fixture.imu_zero_gyro
 6/36 Test  #6: ACheck_Test_Fixture.imu_zero_gyro .......................  Passed
      Start  7: ACheck_Test_Fixture.imu_zero_acc
 7/36 Test  #7: ACheck_Test_Fixture.imu_zero_acc ........................  Passed
      Start  8: ACheck_Test_Fixture.set_quaternion
 8/36 Test  #8: ACheck_Test_Fixture.set_quaternion ......................  Passed
      Start  9: ACheck_Test_Fixture.set_get_gain
 9/36 Test  #9: ACheck_Test_Fixture.set_get_gain ........................  Passed
      Start 10: ACheck_Test_Fixture.get_initial_orientation_imu
10/36 Test #10: ACheck_Test_Fixture.get_initial_orientation_imu .........  Passed
      Start 11: ACheck_Test_Fixture.get_initial_orientation_marg
11/36 Test #11: ACheck_Test_Fixture.get_initial_orientation_marg ........  Passed
      Start 12: ACheck_Estimator_Test_Fixture.update_marg_with_intitial
12/36 Test #12: ACheck_Estimator_Test_Fixture.update_marg_with_intitial .  Passed
      Start 13: ACheck_Estimator_Test_Fixture.update_imu_with_intitial
13/36 Test #13: ACheck_Estimator_Test_Fixture.update_imu_with_intitial ..  Passed
      Start 14: quat_test_suite.invalid_init
14/36 Test #14: quat_test_suite.invalid_init ............................  Passed
      Start 15: quat_test_suite.conjugate_f
15/36 Test #15: quat_test_suite.conjugate_f .............................  Passed
      Start 16: quat_test_suite.conjugate_d
16/36 Test #16: quat_test_suite.conjugate_d .............................  Passed
      Start 17: quat_test_suite.product
17/36 Test #17: quat_test_suite.product .................................  Passed
      Start 18: quat_test_suite.norm_d
18/36 Test #18: quat_test_suite.norm_d ..................................  Passed
      Start 19: quat_test_suite.norm_f
19/36 Test #19: quat_test_suite.norm_f ..................................  Passed
      Start 20: quat_test_suite.scalar_f
20/36 Test #20: quat_test_suite.scalar_f ................................  Passed
      Start 21: quat_test_suite.add_f
21/36 Test #21: quat_test_suite.add_f ...................................  Passed
      Start 22: quat_test_suite.subtract_f
22/36 Test #22: quat_test_suite.subtract_f ..............................  Passed
      Start 23: quat_test_suite.subtract_equals_f
23/36 Test #23: quat_test_suite.subtract_equals_f .......................  Passed
      Start 24: quat_test_suite.set_f
24/36 Test #24: quat_test_suite.set_f ...................................  Passed
      Start 25: quat_test_suite.to_array
25/36 Test #25: quat_test_suite.to_array ................................  Passed
      Start 26: utilities_test_suite.euler_d
26/36 Test #26: utilities_test_suite.euler_d ............................  Passed
      Start 27: utilities_test_suite.euler_f
27/36 Test #27: utilities_test_suite.euler_f ............................  Passed
      Start 28: utilities_test_suite.euler1_f
28/36 Test #28: utilities_test_suite.euler1_f ...........................  Passed
      Start 29: utilities_test_suite.euler2_f
29/36 Test #29: utilities_test_suite.euler2_f ...........................  Passed
      Start 30: utilities_test_suite.rotm_d
30/36 Test #30: utilities_test_suite.rotm_d .............................  Passed
      Start 31: utilities_test_suite.rotm_f
31/36 Test #31: utilities_test_suite.rotm_f .............................  Passed
      Start 32: initializers_test_suite.acc_d
32/36 Test #32: initializers_test_suite.acc_d ...........................  Passed
      Start 33: initializers_test_suite.acc1_f
33/36 Test #33: initializers_test_suite.acc1_f ..........................  Passed
      Start 34: initializers_test_suite.acc2_f
34/36 Test #34: initializers_test_suite.acc2_f ..........................  Passed
      Start 35: initializers_test_suite.mag_d
35/36 Test #35: initializers_test_suite.mag_d ...........................  Passed
      Start 36: initializers_test_suite.mag1_f
36/36 Test #36: initializers_test_suite.mag1_f ..........................  Passed

100% tests passed, 0 tests failed out of 36
\end{verbatim}
\end{small}

\section{Changes Due to Testing}

Two bugs were caught when creating unit tests for the \progname{} module, they were arithmatic
errors that were corrected on the spot. Furthermore, during this process 2 bugs were found in a
popular open source repository. See:
\begin{itemize}
  \item \url{https://github.com/Mayitzin/ahrs/issues/111}
  \item \url{https://github.com/Mayitzin/ahrs/issues/112}
\end{itemize}

\section{Automated Testing}

The unit tests are setup to run automatically when a Pull Request is opened, and after any commit is
made to the $\mathtt{main}$ branch. The
\href{https://github.com/adrian-soch/attitude_check/blob/main/.github/workflows/ci.yaml}{GitHub
workflow} runs the same command as the $\mathtt{build.sh}$ script that is used locally to build and
run tests.

\section{Trace to Requirements}

Table \ref{tab:req_trace} shows the traceability between tests and requirements.
\begin{table}[H]
  \centering
  \caption{Relation of Test Cases to Requirements.}
  \vspace{2mm}
  \label{tab:req_trace}
  \begin{tabular}{|l|l|l|l|l|l|l|l|l|l|l|}
  \hline
     & R1 & R2 & R3 & R4 & R5 & NFR1 & NFR2 & NFR3 & NFR4 & NFR5 \\ \hline
  T1 & X  &    &    &    & X  &      &      &      &      &      \\ \hline
  T2 & X  & X  & X  &    & X  & X    &      &      &      &      \\ \hline
  T3 & X  & X  &    & X  & X  & X    &      &      &      &      \\ \hline
  T4 &    &    &    &    &    & X    &      &      &      &      \\ \hline
  T5 &    &    &    &    &    &      & X    &      &      &      \\ \hline
  T6 &    &    &    &    &    &      &      & X    &      &      \\ \hline
  T7 &    &    &    &    &    &      &      &      & X    &      \\ \hline
  T8 &    &    &    &    &    &      &      &      &      & X    \\ \hline
  \end{tabular}
\end{table}

\section{Trace to Modules}

Table \ref{tab:mod_trace} shows the traceability between tests and modules.

\begin{table}[H]
  \centering
  \caption{Relation of Test Cases to Modules.}
  \vspace{2mm}
  \label{tab:mod_trace}
  \begin{tabular}{|l|l|l|l|l|l|l|l|}
  \hline
     & M0 & M1 & M2 & M3 & M4 & M5 & M6 \\ \hline
  T1 &    &    &    &    &    & X  &    \\ \hline
  T2 &    &    &    &    &    & X  &    \\ \hline
  T3 &    &    &    &    &    & X  &    \\ \hline
  T4 & X  & X  & X  & X  & X  & X  &    \\ \hline
  T5 & X  & X  & X  & X  & X  & X  &    \\ \hline
  T6 &    &    &    &    &    & X  & X  \\ \hline
  T7 &    &    &    &    &    &    &    \\ \hline
  T8 &    &    &    &    &    &    & X  \\ \hline
  \end{tabular}
  \end{table}

\section{Code Coverage Metrics}

The CI pipeline automatically uploads test coverage on the $\mathtt{main}$ branch here:
\url{https://app.codecov.io/gh/adrian-soch/attitude_check/tree/main/src} (scroll to the bottom and
click on individual files). Test coverage is 100\% for all 5 files.

% \bibliographystyle{plainnat}
% \bibliography{../refs/References}

\end{document}
