\documentclass[12pt, titlepage]{article}

\usepackage{amsmath, mathtools}

\usepackage[round]{natbib}
\usepackage{amsfonts}
\usepackage{amssymb}
\usepackage{graphicx}
\usepackage{colortbl}
\usepackage{xr}
\usepackage{hyperref}
\usepackage{longtable}
\usepackage{xfrac}
\usepackage{tabularx}
\usepackage{float}
\usepackage{siunitx}
\usepackage{booktabs}
\usepackage{multirow}
\usepackage[section]{placeins}
\usepackage{caption}
\usepackage{fullpage}

\hypersetup{ bookmarks=true,     % show bookmarks bar?
colorlinks=true,       % false: boxed links; true: colored links
linkcolor=red,          % color of internal links (change box color with linkbordercolor)
citecolor=blue,      % color of links to bibliography
filecolor=magenta,  % color of file links
urlcolor=cyan          % color of external links
}

\usepackage{array}

\externaldocument{../../SRS/SRS}

\input{../../Comments}
%% Common Parts

\newcommand{\progname}{ProgName} % PUT YOUR PROGRAM NAME HERE
\newcommand{\authname}{Adrian Sochaniwsky} % AUTHOR NAMES                  

\usepackage{hyperref}
    \hypersetup{colorlinks=true, linkcolor=blue, citecolor=blue, filecolor=blue,
                urlcolor=blue, unicode=false}
    \urlstyle{same}
                                


\begin{document}

\title{Module Interface Specification for \progname{}}

\author{\authname}

\date{\today}

\maketitle

\pagenumbering{roman}

\section{Revision History}

\begin{tabularx}{\textwidth}{p{3cm}p{2cm}X} \toprule {\bf Date} & {\bf Version} & {\bf Notes}\\
\midrule
\today 1 & 1.0 & Initial document\\
\bottomrule
\end{tabularx}

~\newpage

\section{Symbols, Abbreviations and Acronyms}

See SRS Documentation at
\url{https://github.com/adrian-soch/attitude_check/blob/main/docs/SRS/SRS.pdf}.

\wss{Also add any additional symbols, abbreviations or acronyms}

\newpage

\tableofcontents

\newpage

\pagenumbering{arabic}

\section{Introduction}

The following document details the Module Interface Specifications for \progname.

Complementary documents include the System Requirement Specifications and Module Guide.  The full
documentation and implementation can be found at
\url{https://github.com/adrian-soch/attitude_check}.  \wss{provide the url for your repo}

\section{Notation}

\wss{You should describe your notation.  You can use what is below as a starting point.}

The structure of the MIS for modules comes from \citet{HoffmanAndStrooper1995}, with the addition
that template modules have been adapted from \cite{GhezziEtAl2003}.  The mathematical notation comes
from Chapter 3 of \citet{HoffmanAndStrooper1995}.  For instance, the symbol := is used for a
multiple assignment statement and conditional rules follow the form $(c_1 \Rightarrow r_1 | c_2
\Rightarrow r_2 | ... | c_n \Rightarrow r_n )$.

The following table summarizes the primitive data types used by \progname.

\begin{center}
\renewcommand{\arraystretch}{1.2}
\noindent
\begin{tabular}{l l p{7.5cm}}
\toprule
\textbf{Data Type} & \textbf{Notation} & \textbf{Description}\\
\midrule
% character & char & a single symbol or digit\\
% integer & $\mathbb{Z}$ & a number without a fractional component in (-$\infty$, $\infty$) \\
% natural number & $\mathbb{N}$ & a number without a fractional component in [1, $\infty$) \\
real & $\mathbb{R}$ & any number in (-$\infty$, $\infty$)\\
boolean & $\mathbb{B}$ & value in $[\text{false}=0, \text{true}=1]$\\
matrix & $\mathbb{R}^{m \times n}$ & matrix of any number in (-$\infty$, $\infty$)\\
vector & $\mathbb{R}^{m}$ & column vector of any number in (-$\infty$, $\infty$)\\
quaternion & $\mathbf{q}$ & a quaternion $\in \mathbb{R}^4$, see SRS for details\\
\bottomrule
\end{tabular}
\end{center}

\noindent
The specification of \progname \ uses some derived data types: sequences, strings, and tuples.
Sequences are lists filled with elements of the same data type. Strings are sequences of characters.
Tuples contain a list of values, potentially of different types. In addition, \progname \ uses
functions, which are defined by the data types of their inputs and outputs. Local functions are
described by giving their type signature followed by their specification.

\section{Module Decomposition}

The following table is taken directly from the Module Guide document for this project.

\begin{table}[h!]
\centering
\begin{tabular}{p{0.3\textwidth} p{0.6\textwidth}}
\toprule
\textbf{Level 1} & \textbf{Level 2}\\
\midrule

% {Hardware-Hiding Module} & ~ \\
% \midrule

\multirow{7}{0.3\textwidth}{Behaviour-Hiding Module} & Control Module \\
& Input Verification Module \\
& Accel To Quat Module \\
& Mag To Quat Module \\
& Estimate w/o Mag Module \\
& Estimate w Mag Module \\
\midrule
\multirow{2}{0.3\textwidth}{Software Decision Module} & Matrix Math Module\\
& Quaternion Module \\
\bottomrule

\end{tabular}
\caption{Module Hierarchy}
\label{TblMH}
\end{table}

\newpage
~\newpage

\section{MIS of Control Module} \label{cm}

\wss{You can reference SRS labels, such as R\ref{R_Inputs}.}

\wss{It is also possible to use \LaTeX for hypperlinks to external documents.}

\subsection{Module}
control

\subsection{Uses}
Input Verification Module \newline
Accel To Quat Module \newline
Mag To Quat Module \newline
Estimate w/o Mag Module \newline
Estimate w Mag Module \newline
Quaternion Module \newline
Math Module \newline
Matrix Math Module

\subsection{Syntax}

\subsubsection{Exported Access Programs}

\begin{center}
\begin{tabular}{p{2cm} p{4cm} p{4cm} p{2cm}}
\hline
\textbf{Name} & \textbf{In} & \textbf{Out} & \textbf{Exceptions} \\
\hline
\wss{accessProg} & - & - & - \\
\hline
\end{tabular}
\end{center}

\subsection{Semantics}

\subsubsection{State Variables}

% \subsubsection{Assumptions}

\subsubsection{Access Routine Semantics}

\noindent \wss{accessProg}():
\begin{itemize}
\item transition: \wss{if appropriate}
\item output: \wss{if appropriate}
\item exception: \wss{if appropriate}
\end{itemize}

\newpage

%%%%%%%%%%%%%%%%%%%%%%%%%%%%%%%%%%%%%%%%%%%%%%%%%%%%%%%%%%%%%%%
%%%%%%%%%%%%%%%%%%%%%%%%%%%%%%%%%%%%%%%%%%%%%%%%%%%%%%%%%%%%%%%
\section{MIS of Estimate w/o Mag Module} \label{ewomm}

\subsection{Module}
estimate\_wo\_mag

\subsection{Uses}
Accel To Quat Module \newline
Quaternion Module \newline
Math Module \newline
Matrix Math Module

\subsection{Syntax}

\subsubsection{Exported Constants}

\subsubsection{Exported Access Programs}

\begin{center}
\begin{tabular}{p{2cm} p{4cm} p{4cm} p{2cm}}
\hline
\textbf{Name} & \textbf{In} & \textbf{Out} & \textbf{Exceptions} \\
\hline
\wss{accessProg} & - & - & - \\
\hline
\end{tabular}
\end{center}

\subsection{Semantics}

\subsubsection{State Variables}

\subsubsection{Access Routine Semantics}

\noindent \wss{accessProg}():
\begin{itemize}
\item transition: \wss{if appropriate}
\item output: \wss{if appropriate}
\item exception: \wss{if appropriate}
\end{itemize}

\newpage

%%%%%%%%%%%%%%%%%%%%%%%%%%%%%%%%%%%%%%%%%%%%%%%%%%%%%%%%%%%%%%%
%%%%%%%%%%%%%%%%%%%%%%%%%%%%%%%%%%%%%%%%%%%%%%%%%%%%%%%%%%%%%%%
\section{MIS of Estimate w Mag Module} \label{ewmm}

\subsection{Module}
estimate\_w\_mag

\subsection{Uses}
Mag To Quat Module \newline
Quaternion Module \newline
Math Module \newline
Matrix Math Module

\subsection{Syntax}

% \subsubsection{Exported Constants}

\subsubsection{Exported Access Programs}

\begin{center}
\begin{tabular}{p{2cm} p{4cm} p{4cm} p{2cm}}
\hline
\textbf{Name} & \textbf{In} & \textbf{Out} & \textbf{Exceptions} \\
\hline
\wss{accessProg} & - & - & - \\
\hline
\end{tabular}
\end{center}

\subsection{Semantics}

\subsubsection{State Variables}

% \subsubsection{Assumptions}

\subsubsection{Access Routine Semantics}

\noindent \wss{accessProg}():
\begin{itemize}
\item transition: \wss{if appropriate}
\item output: \wss{if appropriate}
\item exception: \wss{if appropriate}
\end{itemize}

% \subsubsection{Local Functions}

\newpage

%%%%%%%%%%%%%%%%%%%%%%%%%%%%%%%%%%%%%%%%%%%%%%%%%%%%%%%%%%%%%%%
%%%%%%%%%%%%%%%%%%%%%%%%%%%%%%%%%%%%%%%%%%%%%%%%%%%%%%%%%%%%%%%
\section{MIS of Accel To Quat Module} \label{iqewomm}

\subsection{Module}
acc2quat

\subsection{Uses}
Quaternion Module \newline
Math Module \newline
Matrix Math Module

\subsection{Syntax}

\subsubsection{Exported Access Programs}

\begin{center}
\begin{tabular}{p{2cm} p{4cm} p{4cm} p{2cm}}
\hline
\textbf{Name} & \textbf{In} & \textbf{Out} & \textbf{Exceptions} \\
\hline
acc\_to\_quat & $\mathbf{a}:=\mathbb{R}^3$ & $q_\text{out} := \mathbf{q}$ & - \\
\hline
\end{tabular}
\end{center}

\subsection{Semantics}

\subsubsection{Access Routine Semantics}

\noindent mag\_to\_quat($\mathbf{a}:=\mathbb{R}^3$):
\begin{itemize}
\item output: $q_\text{out}:= \text{euler\_to\_quat}(\theta, \phi, \psi)$ given
\begin{align*}
  \theta &= \text{atan2}(a_y, a_z) \\
  \phi &= \text{atan2}(-a_x, \sqrt{a_y^2+a_z^2}) \\
  \psi &= 0
\end{align*}
where $[a_x, a_y, a_z] = \cfrac{\mathbf{a}}{|\mathbf{a}|}$
\item exception: none
\end{itemize}

\newpage

%%%%%%%%%%%%%%%%%%%%%%%%%%%%%%%%%%%%%%%%%%%%%%%%%%%%%%%%%%%%%%%
%%%%%%%%%%%%%%%%%%%%%%%%%%%%%%%%%%%%%%%%%%%%%%%%%%%%%%%%%%%%%%%
\section{MIS of Mag To Quat Module} \label{iqewmm}

\subsection{Module}
mag2quat

\subsection{Uses}
Quaternion Module \newline
Matrix Math Module \newline
Math Module

\subsection{Syntax}

\subsubsection{Exported Access Programs}

\begin{center}
\begin{tabular}{p{3cm} p{4cm} p{3cm} p{2cm}}
\hline
\textbf{Name} & \textbf{In} & \textbf{Out} & \textbf{Exceptions} \\
\hline
mag\_to\_quat & $\mathbf{m}:=\mathbb{R}^3, \mathbf{a}:=\mathbb{R}^3$ & $q_\text{out} := \mathbf{q}$ & ValueError \\
\hline
\end{tabular}
\end{center}

\subsection{Semantics}

\subsubsection{Access Routine Semantics}

\noindent mag\_to\_quat($\mathbf{m}:=\mathbb{R}^3, \mathbf{a}:=\mathbb{R}^3$):
\begin{itemize}
\item output: $q_\text{out}:= \text{euler\_to\_quat}(\theta, \phi, \psi)$ given
\begin{align*}
  \theta &= \text{atan2}(a_y, a_z) \\
  \phi &= \text{atan2}(-a_x, \sqrt{a_y^2+a_z^2}) \\
  \psi &= \text{atan2}(m_z\sin\phi - m_y\cos\phi, m_x\cos\theta + \sin\theta(m_y\sin\phi + m_z\cos\phi))
\end{align*}
where $[a_x, a_y, a_z] = \cfrac{\mathbf{a}}{|\mathbf{a}|}$ and $[m_x, m_y, m_z] = \cfrac{\mathbf{m}}{|\mathbf{m}|}$
\item exception: ValueError
\end{itemize}

\newpage

%%%%%%%%%%%%%%%%%%%%%%%%%%%%%%%%%%%%%%%%%%%%%%%%%%%%%%%%%%%%%%%
%%%%%%%%%%%%%%%%%%%%%%%%%%%%%%%%%%%%%%%%%%%%%%%%%%%%%%%%%%%%%%%
\section{MIS of Input Verification Module} \label{ivm}

\subsection{Module}
input

\subsection{Uses}
None

\subsection{Syntax}

\subsubsection{Exported Access Programs}

\begin{center}
\begin{tabular}{p{3cm} p{8cm} p{1cm} p{2cm}}
\hline
\textbf{Name} & \textbf{In} & \textbf{Out} & \textbf{Exceptions} \\
\hline
input\_checker & $q:=\mathbf{q}, \text{gyr}:=\mathbb{R}^3, \text{acc}:=\mathbb{R}^3, \text{mag}:=\mathbb{R}^3, \text{dt}:=\mathbb{R}$ & y:=$\mathbb{B}$ & ValueError \\
input\_checker & $q:=\mathbf{q}, \text{gyr}:=\mathbb{R}^3, \text{acc}:=\mathbb{R}^3, \text{dt}:=\mathbb{R}$ & y:=$\mathbb{B}$ & ValueError \\
\hline
\end{tabular}
\end{center}

\subsection{Semantics}

% \subsubsection{State Variables}

\subsubsection{Access Routine Semantics}

\noindent input\_checker($q:=\mathbf{q}, \text{gyr}:=\mathbb{R}^3, \text{acc}:=\mathbb{R}^3, \text{mag}:=\mathbb{R}^3, \text{dt}:=\mathbb{R}$ ):
\begin{itemize}
\item output: y:= true, if input values are within the bounds specified in Section 4.2.8 Input Data Constraints of the SRS.
\item exception: ValueError
\end{itemize}

\noindent input\_checker($q:=\mathbf{q}, \text{gyr}:=\mathbb{R}^3, \text{acc}:=\mathbb{R}^3, \text{dt}:=\mathbb{R}$):
\begin{itemize}
\item output: y:= true, if input values are within the bounds specified in Section 4.2.8 Input Data Constraints of the SRS.
\item exception: ValueError
\end{itemize}

\newpage

%%%%%%%%%%%%%%%%%%%%%%%%%%%%%%%%%%%%%%%%%%%%%%%%%%%%%%%%%%%%%%%
%%%%%%%%%%%%%%%%%%%%%%%%%%%%%%%%%%%%%%%%%%%%%%%%%%%%%%%%%%%%%%%
\section{MIS of Quaternion Module} \label{qm}

\subsection{Module}
quaternion

\subsection{Uses}
Matrix Math Module \newline
Math Module

\subsection{Syntax}

\subsubsection{Exported Access Programs}

\begin{center}
\begin{tabular}{p{2.5cm} p{6cm} p{2cm} p{1.5cm}}
\hline
\textbf{Name} & \textbf{In} & \textbf{Out} & \textbf{Exceptions} \\
\hline
create\_quat & w:=$\mathbb{R}, x:=\mathbb{R}, y:=\mathbb{R}, z:=\mathbb{R}$ & - & ValueError \\
quat\_prod & $p:=\mathbf{q},  q:=\mathbf{q}$ & $q_\text{out}:=\mathbf{q}$ & ValueError \\
normalize & - & - & ValueError \\
assert\_is\_norm & w:=$\mathbb{R}, x:=\mathbb{R}, y:=\mathbb{R}, z:=\mathbb{R}$ & out$:=\mathbb{B}$
& ValueError \\
quat\_to\_euler & - & $\mathbf{e}:=\mathbb{R}^3$ & ValueError \\
quat\_to\_rot & - & $\mathbf{R} := \mathbb{R}^{3 \times 3}$ & ValueError \\
\hline
\end{tabular}
\end{center}

\subsection{Semantics}

\subsubsection{State Variables}

$\text{quat}: \mathbf{q}$

\subsubsection{Access Routine Semantics}

\noindent create\_quat($w, x, y, z$):
\begin{itemize}
\item transition: quat $:= \mathbf{q}$ where $\mathbf{q} = [w, x, y, z]$
% \item output: $\t\mathbf{q}$
\item exception: ValueError when  $|\text{quat}| \neq 1$
\end{itemize}

\noindent quat\_prod($p, q$):
\begin{itemize}
% \item transition: \wss{if appropriate}
\item output:
  \begin{align*}
    q_\text{out} :=  \begin{bmatrix}
      p_w q_w - p_x q_x - p_y q_y - p_z q_z \\
      p_w q_x + p_x q_w + p_y q_z - p_z q_y \\
      p_w q_y - p_x q_z + p_y q_w + p_z q_x \\
      p_w q_z + p_x q_y - p_y q_x + p_z q_w
  \end{bmatrix}
  \end{align*}

\item exception: ValueError
\end{itemize}

\noindent normalize():
\begin{itemize}
\item transition: quat $:= \left[ \cfrac{\text{quat}_w}{d}, \cfrac{\text{quat}_x}{d},
\cfrac{\text{quat}_y}{d}, \cfrac{\text{quat}_z}{d} \right]$ where $d = \sqrt{\text{quat}_w^2 +
\text{quat}_x^2 + \text{quat}_y^2 + \text{quat}_z^2}$
% \item output: $\mathbf{q}$
\item exception: ValueError
\end{itemize}

\noindent assert\_is\_norm():
\begin{itemize}
\item output: out$:= (1 == \sqrt{w^2 + x^2 + y^2 + z^2})$
\item exception: ValueError
\end{itemize}

\noindent quat\_to\_euler():
\begin{itemize}
\item output: \begin{align*}
  \mathbf{e} :=& \begin{bmatrix}
    \text{yaw} \\ \text{pitch} \\ \text{roll}
  \end{bmatrix} =
  \begin{bmatrix}
    \text{atan2}(2 q_y q_w - 2 q_x q_z, 1 - 2 q_y^2 - 2 q_z ^ 2) \\
    \text{asin}(2 q_x q_y + 2 q_z q_w) \\
    \text{atan2}(2 q_x q_w - 2 q_y q_z, 1 - 2 q_x^2 - 2 q_z ^2)
  \end{bmatrix}
\end{align*}
See \url{https://www.euclideanspace.com/maths/geometry/rotations/conversions/quaternionToEuler/index.htm} for 2 special cases.
\item exception: ValueError
\end{itemize}

\noindent quat\_to\_rot():
\begin{itemize}
\item output: \begin{align*}
  \mathbf{R}:= & \begin{bmatrix}
    1- 2 q_y^2 - 2 q_z^2 & 2 q_x q_y - 2 q_z q_w & 2 q_x q_z + 2 q_y q_w \\
    2 q_x q_y + 2 q_z q_w & 1 -2 q_x^2 - 2 q_z^2 & 2 q_y q_z - 2 q_x q_w \\
    2 q_x q_z - 2 q_y q_w & 2 q_y q_z + 2 q_x q_w & 1 -2 q_x^2 - 2 q_y ^2
  \end{bmatrix}
\end{align*}
\item exception: ValueError
\end{itemize}

\newpage

%%%%%%%%%%%%%%%%%%%%%%%%%%%%%%%%%%%%%%%%%%%%%%%%%%%%%%%%%%%%%%%
%%%%%%%%%%%%%%%%%%%%%%%%%%%%%%%%%%%%%%%%%%%%%%%%%%%%%%%%%%%%%%%
\section{MIS of Matrix Math Module} \label{mmm}

\subsection{Module}
matrix

\subsection{Uses}
N/A

\subsection{Syntax}

\subsubsection{Exported Access Programs}

\begin{center}
\begin{tabular}{p{2cm} p{3cm} p{2.5cm} p{4.5cm}}
\hline
\textbf{Name} & \textbf{In} & \textbf{Out} & \textbf{Exceptions} \\
\hline
* & $\mathbb{R}^{m \times n} \times \mathbb{R}^{n \times m}$ & $m:=\mathbb{R}^{n \times n}$ &
ValueError \\
* & $\mathbb{R}^{m \times n} \times \mathbb{R}$ & $m:=\mathbb{R}^{m \times n}$ &
ValueError \\
+ & $\mathbb{R}^{m \times n} \times \mathbb{R}^{m \times n}$ & $m:=\mathbb{R}^{m \times n}$ &
ValueError \\
% - & $\mathbb{R}^{m \times n} \times \mathbb{R}^{m \times n}$ & $\mathbb{R}^{m \times n}$ &
%   ValueError \\
transpose & $\mathbb{R}^{m \times n}$ & $m:=\mathbb{R}^{n \times m}$ & ValueError
\\
norm & $\mathbf{x}:=\mathbb{R}^n$ & $\mathbf{y}:=\mathbb{R}^n$ & ValueError\\
\hline
\end{tabular}
\end{center}

\subsection{Semantics}

% \subsubsection{State Variables}

% None.

\subsubsection{Access Routine Semantics}

\noindent transpose($\mathbb{R}^{m \times n}$):
\begin{itemize}
% \item transition: \wss{if appropriate}
\item output: $m:=\mathbb{R}^{m \times n}$
\item exception: ValueError
\end{itemize}

$[\mathbf{A}^T]_{i,j} = [\mathbf{A}]_{j,i}$
\newline

\noindent $\mathbb{R}^{m \times n} * \mathbb{R}^{n \times m}$:
\begin{itemize}
\item output: $m:=\mathbb{R}^{n \times n}$
\item exception: ValueError
\end{itemize}

Let $\mathbf{A} = [a_{i,j}]_{m\times n}$ and $\mathbf{B} = [b_{i,j}]_{n\times m}$. Then $\mathbf{C}
= \mathbf{A} * \mathbf{B}$ with $c_{i,j} = a_{i,0} b_{0,j} + a_{i,1} b_{1,j} ...  a_{i,n}b_{n,j}$.
\newline

\noindent $\mathbb{R}^{m \times n} * \mathbb{R}$:
\begin{itemize}
\item output: $m:=\mathbb{R}^{m \times n}$
\item exception: ValueError
\end{itemize}

Let $\mathbf{A} = [a_{i,j}]_{m\times n}$ and $k = \mathbb{R}$. Then $\mathbf{C} = \mathbf{A} * k$
with $c_{i,j} = k a_{i,j}$.
\newline

\noindent $\mathbb{R}^{m \times n} + \mathbb{R}^{m \times n}$:
\begin{itemize}
\item output: $m:=\mathbb{R}^{m \times n}$
\item exception: ValueError
\end{itemize}

Let $\mathbf{A} = [a_{i,j}]_{m \times n}$ and $\mathbf{B} = [b_{i,j}]_{m \times n}$. Then
$\mathbf{A} + \mathbf{B} = [a_{i,j} + b_{i,j}]_{m \times n}$.
\newline

\noindent norm(x:=$\mathbb{R}^{n}$):
\begin{itemize}
\item output: $y:=\mathbb{R}^{n}$ where $y = \cfrac{\mathbf{x}}{|\mathbf{x}|}$
\item exception: ValueError
\end{itemize}

\newpage

\section{MIS of Math Module} \label{mm}

\subsection{Module}
math

\subsection{Uses}
None

\subsection{Syntax}

\subsubsection{Exported Constants}
\begin{itemize}
  \item[PI] := 3.141592654
  \item[RAD2DEG] := $\cfrac{180}{\text{PI}}$
  \item[DEG2RAD] := $\cfrac{\text{PI}}{180}$
\end{itemize}


\subsubsection{Exported Access Programs}

\begin{center}
\begin{tabular}{p{2cm} p{4cm} p{4cm} p{2cm}}
\hline
\textbf{Name} & \textbf{In} & \textbf{Out} & \textbf{Exceptions} \\
\hline
sin & x:=$\mathbb{R}$ & out:=$\mathbb{R}$ & ValueError \\
cos & x:=$\mathbb{R}$ & out:=$\mathbb{R}$ & ValueError \\
asin & x:=$\mathbb{R}$ & out:=$\mathbb{R}$ & ValueError \\
atan2 & x:=$\mathbb{R}$, y:=$\mathbb{R}$ & out:=$\mathbb{R}$ & ValueError \\
euler\_to\_quat & $\mathbf{e}:=\mathbb{R}^3$& $y:=\mathbf{q}$ & ValueError \\
rot\_to\_quat & $\mathbf{R} := \mathbb{R}^{3 \times 3}$ & $y:=\mathbf{q}$ & ValueError \\
\hline
\end{tabular}
\end{center}

\subsection{Semantics}

\subsubsection{Access Routine Semantics}

\noindent sin(x:=$\mathbb{R}$):
\begin{itemize}
\item output: $y = \sin(x)$
\item exception: ValueError
\end{itemize}

\noindent cos(x:=$\mathbb{R}$):
\begin{itemize}
\item output: $y = \cos(x)$
\item exception: ValueError
\end{itemize}

\noindent asin(x:=$\mathbb{R}$):
\begin{itemize}
\item output: $y = \sin(x)$
\item exception: ValueError
\end{itemize}

\noindent atan2(y:=$\mathbb{R}$, x:=$\mathbb{R}$):
\begin{itemize}
\item output: See \url{https://en.wikipedia.org/wiki/Atan2#Definition_and_computation}
\item exception: ValueError if $x =0, y =0$
\end{itemize}

\noindent euler\_to\_quat($\mathbf{e}:=\mathbb{R}^3$):
\begin{itemize}
\item output: See \url{https://www.euclideanspace.com/maths/geometry/rotations/conversions/eulerToQuaternion/index.htm}
\item exception: ValueError
\end{itemize}

\noindent rot\_to\_quat($\mathbf{R} := \mathbb{R}^{3 \times 3}$):
\begin{itemize}
\item output: See \url{https://www.euclideanspace.com/maths/geometry/rotations/conversions/matrixToQuaternion/}
\item exception: ValueError
\end{itemize}

\newpage

\bibliographystyle {plainnat}
\bibliography {../../../refs/References}

% \newpage \section{Appendix} \label{Appendix}
\end{document}
